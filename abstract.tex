%!TEX root = proposal.tex
\begin{abstract}
%\textcolor{blue}{
%The thesis proposal is a type of contract between the faculty and the student. 
%An accepted thesis proposal indicates that the work proposed by the student, 
%once completed, will be accepted by the faculty as sufficiently innovative and 
%substantial as to be recognized with the award of the degree. It is part of 
%the training of the student's research apprenticeship that the form of this 
%proposal must be as concise as those proposals required by major funding 
%agencies.
%The abstract is a brief summary of your Ph.D. Research Proposal, and 
%should be no longer than 200 words. It starts by describing in a few 
%words the knowledge domain where your research takes place and the 
%key issues of that domain that offer opportunities for the scientific or 
%technological innovations you intend to explore. Taking those key issues 
%as a background, you then present briefly your research statement, your 
%proposed research approach, the results you expect to achieve, and the 
%anticipated implications of such results on the advancement of the 
%knowledge domain. 
%To keep your abstract concise and objective, imagine that you were 
%looking for financial support from someone who is very busy. Suppose 
%that you were to meet that person at an official reception and that she 
%would be willing to listen to you for no more than two minutes. What 
%you would say to that person, and the pleasant style you would adopt in 
%those two demanding minutes, is what you should put in your abstract. 
%The guidelines provided in this template are meant to be used creatively 
%and not, by any means, as a cookbook recipe for the production of 
%research proposals. 
%}

Detecting rocky planets orbiting inside the habitable zones of their host stars is one of the main goals of planetary research. These are at the limit of current observational capabilities, but the number of super-Earths and Neptune-like planets has been growing steadily. Planet formation models indicate that lighter planets should be frequent around low metallicity stars, contrarily to what is observed for giant planets. Considering this, we aim to explore data from an ongoing survey (an ESO Large Program) being performed by the planet group in CAUP, which uses the HARPS spectrograph to search for low-mass exoplanets around 109 solar-type, moderately metal-poor stars. We will study different data analysis methods, aiming at developing and perfecting the tools necessary to detect very low mass planets through radial velocity measurements and expect to characterize their frequency distribution down to the Earth-mass limit, in the metal-poor regime. Finally, this project will help us create the necessary know-how to analyze data coming from existing and future spectrographs.

\titleformat{\subsubsection}[block]
{\normalfont\bfseries\filcenter}{\fbox{\itshape\thesubsubsection}}{1em}{}

\subsubsection*{\centering Keywords}
%This section is an alphabetically ordered list of the more appropriate 
%words or expressions (up to twelve) that you would introduce in a search 
%engine to find a research proposal identical to yours. The successive 
%keywords are separated by comas
planetary systems – exoplanets – Stars: solar-type – Stars: metal-poor –
Methods: data analysis – Methods: statistical – Techniques: radial velocities – 
Catalogs


\end{abstract}